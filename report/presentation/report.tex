%Please use LuaLaTeX or XeLaTeX
\documentclass[11pt,aspectratio=169]{beamer}

\title{Optimization Methods for Engineers}
\date[July 2023]{Spring Semester 2023}
\author{Thierry Meier, Geventh *NAME*}
\institute{Optimization Methods for Engineers\\227-0707-00L}

\usetheme{eth}

\colorlet{titlefgcolor}{ETHBlue}
\colorlet{accentcolor}{ETHRed}

\usepackage{caption}
\usepackage{transparent}

\definecolor{tabblue}{RGB}{31,119,180}
\colorlet{tabbluealpha}{tabblue!25}

\begin{document}

\def\titlefigure{elements/eggholder.pdf}        % Default image

%
% SLIDE
%

\titleframe

%
% SLIDE
%

\tocframe

%
% SLIDE
%

\section{Project Goal}

%
% SLIDE
%

\begin{frame}[fragile]{Project Goal}
    \begin{columns}
    \begin{column}{0.5\textwidth}


        \begin{itemize}
            \item \textbf{understand} and \textbf{implement} heuristics behind SA and PSO
            \item \textbf{develop} small, modular python optimization library, called \href{https://github.com/thmeier}{\textbf{sapso}}
            \item \textbf{propose new optimizer}, based on SA and PSO to overcome their limitations
            \item tune hyperparameters by \textbf{sensitivity study}
            \item \textbf{compare} optimizers via benchmarking
        \end{itemize}

    \end{column}
    \begin{column}{0.5\textwidth}
        %SA and PSO were chosen because we both are amazed how these algorithms, based
        Why we like and chose SA and PSO
        \begin{itemize}
            \item nature-inspired
            \item gradient free
            \item simple yet elegant ideas
        \end{itemize}
    \end{column}
    \end{columns}
\end{frame}

%
% SLIDE
%

\section{Adaptive/Annealing Particle Swarm Optimization}

%
% SLIDE
%

\begin{frame}[fragile]{APSO = SA $\times$ PSO}
    \begin{columns}
    \begin{column}{0.45\textwidth}
        \begin{itemize}
            \item APSO $=$ Adaptive PSO
            \item inspiration from SA
            \begin{itemize}
                \item \textbf{dynamically adapt} internal \textbf{parameters} as iterations progress
            \end{itemize}
            \item inspiration from PSO
            \begin{itemize}
                \item \textbf{multiple agents} for co-joint search space exploration
            \end{itemize}
            \item idea for APSO:
            \begin{itemize}
                \item \textbf{divide movement} of PSO agents into \textbf{phases} by \textbf{adaptively weighting} the velocity update \textbf{components} w.r.t. iterations
            \end{itemize}
        \end{itemize}
    \end{column}

    \begin{column}{0.6\textwidth}
        \begin{verbatim}
        # PSO velocity update rule
        velocity = (
            # inertial component
            w * velocity
            # self best component
            + a_self * r_self * dir_best_self
            # swarm best component
            + a_swarm * r_swarm * dir_best_swarm
        )
        \end{verbatim}
    \end{column}
    \end{columns}
\end{frame}

%
% SLIDE
%

\begin{frame}[fragile]{APSO = SA $\times$ PSO}
    \begin{columns}
    \begin{column}{0.5\textwidth}

        \begin{itemize}
            \item \textbf{phase 0}: global exploration phase\\
                {\scriptsize{\textit{Agents explore the optimization landscape individually while not being drawn to one another.}}}
            \item \textbf{phase 1}: local exploration phase\\
                {\scriptsize{\textit{Agents focus on exploring their own best encountered optimum.}}}

            \item \textbf{phase 2}: optimum finding phase\\
                {\scriptsize{\textit{All agents focus on exploring the best encountered swarm optimum.}}}
            \smallskip
            \item velocity update components
            \begin{itemize}
                \item \textbf{component 0}: inertial component
                \item \textbf{component 1}: self best component
                \item \textbf{component 2}: swarm best component
            \end{itemize}
            %\smallskip
            %\item APSO algorithm:
            %\begin{itemize}
            %    \item PSO except with weighting of each component by the respective cubic polynomial
            %\end{itemize}
        \end{itemize}
    \end{column}

    \begin{column}{0.45\textwidth}
        \includegraphics[width=\columnwidth]{../plots/apso_velocity_component_weighting__const.pdf}
        \captionof{ }{\tiny{Constant adaptive velocity component weighting.}}
    \end{column}
    \end{columns}
\end{frame}

%
% SLIDE
%

\begin{frame}[fragile]{APSO = SA $\times$ PSO}
    \begin{columns}
    \begin{column}{0.5\textwidth}

        \begin{itemize}
            \item \textbf{phase 0}: global exploration phase\\
                {\scriptsize{\textit{Agents explore the optimization landscape individually while not being drawn to one another.}}}
            \item \textbf{phase 1}: local exploration phase\\
                {\scriptsize{\textit{Agents focus on exploring their own best encountered optimum.}}}

            \item \textbf{phase 2}: optimum finding phase\\
                {\scriptsize{\textit{All agents focus on exploring the best encountered swarm optimum.}}}
            \smallskip
            \item velocity update components
            \begin{itemize}
                \item \textbf{component 0}: inertial component
                \item \textbf{component 1}: self best component
                \item \textbf{component 2}: swarm best component
            \end{itemize}
            %\smallskip
            %\item APSO algorithm:
            %\begin{itemize}
            %    \item PSO except with weighting of each component by the respective cubic polynomial
            %\end{itemize}
        \end{itemize}
    \end{column}

    \begin{column}{0.45\textwidth}
        \includegraphics[width=\columnwidth]{../plots/apso_velocity_component_weighting__cubic.pdf}
        \captionof{ }{\tiny{Cubic adaptive velocity component weighting.}}
    \end{column}
    \end{columns}
\end{frame}

%
% SLIDE
%

\begin{frame}[fragile]{APSO = SA $\times$ PSO}
    \begin{columns}
    \begin{column}{0.5\textwidth}

        \begin{itemize}
            \item \textbf{phase 0}: global exploration phase\\
                {\scriptsize{\textit{Agents explore the optimization landscape individually while not being drawn to one another.}}}
            \item \textbf{phase 1}: local exploration phase\\
                {\scriptsize{\textit{Agents focus on exploring their own best encountered optimum.}}}

            \item \textbf{phase 2}: optimum finding phase\\
                {\scriptsize{\textit{All agents focus on exploring the best encountered swarm optimum.}}}

            \smallskip
            \item velocity update components
            \begin{itemize}
                \item \textbf{component 0}: inertial component
                \item \textbf{component 1}: self best component
                \item \textbf{component 2}: swarm best component
            \end{itemize}
            %\smallskip
            %\item APSO algorithm:
            %\begin{itemize}
            %    \item PSO except with weighting of each component by the respective cubic polynomial
            %\end{itemize}
        \end{itemize}
    \end{column}

    \begin{column}{0.45\textwidth}
        \includegraphics[width=\columnwidth]{../plots/apso_velocity_component_weighting__exp.pdf}
        \captionof{ }{\tiny{Exponential adaptive velocity component weighting.}}
    \end{column}
    \end{columns}
\end{frame}

%
% SLIDE
%

\section{Test Functions}

%
% SLIDE
%

\begin{frame}[fragile]{Test Functions - Overview}
    \begin{itemize}
        \item taken from \href{https://en.wikipedia.org/wiki/Test_functions_for_optimization}{\textbf{Wikipedia}}, 
            \href{https://www.sfu.ca/~ssurjano/michal.html}{\textbf{Virtual Library of Simulation Experiments}}
        \item characteristically different minimization problems
            \begin{itemize}
                \item ill conditioned $\implies$ Schaffner, Bukin
                \item high-frequency landscape $\implies$ Ackley, Eggholder, H\"older Table
                \item valley-like landscape $\implies$ Michalewicz, Bukin
            \end{itemize}

    \end{itemize}
\end{frame}

%
% SLIDE
%

\begin{frame}[fragile]{Test Functions - Contour Plot}

    \begin{columns}
    \begin{column}{0.8\textwidth}
        \includegraphics[width=\columnwidth]{../plots/overview_testfunction_contour.pdf}
    \end{column}
    \end{columns}
    \tiny{Note: White plus symbols denote global minima}

\end{frame}

%
% SLIDE
%

\begin{frame}[fragile]{Test Functions - Surface Plot}

    \begin{columns}
    \begin{column}{0.8\textwidth}
        \includegraphics[width=\columnwidth]{../plots/overview_testfunction_surface.pdf}
    \end{column}
    \end{columns}

    \tiny{Note: best viewed in full-screen!}

\end{frame}

%
% SLIDE
%

\section{Sensitivity Study}

%
% SLIDE
%

\begin{frame}[fragile]{Sensitivity Study - Overview}

    Initial Hyperparam Search for \textbf{SA, PSO, APSO}
    \begin{enumerate}
        \item \textbf{estimate default} parameters by intuition and playing around
        \item \textbf{assess sensitivity} of each parameter individually\\
            $\implies$ new default parameters by visual inspection
            \begin{itemize}
                \item perform $10$ stochastically \textbf{independent retries} of same experiments\\
                    $\implies$ reduce fluctuations due to randomness and noise
                \item track \textbf{ratio} of optimum \textbf{found vs. best}\\
                    $\implies$ unify comparison of optimizers on different test functions
            \end{itemize}
        \item \textbf{assess sensitivity} of each parameter with new defaults \textbf{again}\\
            $\implies$ defaults for benchmark
    \end{enumerate}
\end{frame}

%
% SLIDE
%

\begin{frame}[fragile]{Sensitivity Study - Parameters \& Initial Defaults}

    \begin{columns}
    \begin{column}{0.5\textwidth}
        \textbf{Parameters of Simulated Annealing}
        \begin{itemize}
            \item number of iterations
            \item step size (in \% relative to search space)
            \item temperature
                \begin{itemize}
                    \item linear decay
                    \item exponential decay
                \end{itemize}
        \end{itemize}

        \medskip % \smallskip % \bigskip

        \textbf{Parameters of Particle Swarm Optimization}
        \begin{itemize}
            \item number of iterations
            \item number of particles
            \item inertial weight
            \item acceleration individual optimum
            \item acceleration swarm optimum
        \end{itemize}
    \end{column}

    \begin{column}{0.5\textwidth}
        \begin{center}
            \begin{tabular}{@{}lrrr@{}}
                \toprule
                & \multicolumn{3}{c}{Defaults}       \\
                \cmidrule(r){2-4}Parameter & SA & PSO & APSO\\
                \midrule
                iterations    & $1000$ &  $250$ &  $500$ \\
                step size     & $10$\% &        &        \\
                temperature   &   exp. &        &        \\
                particles     &        &  $100$ &  $100$ \\
                inertia       &        & $0.75$ & $0.75$ \\
                individual    &        &  $1.0$ &  $1.0$ \\
                swarm         &        &  $1.6$ &  $1.6$ \\
                interpolation &        &        & const. \\
                \bottomrule
            \end{tabular}
            \captionof{ }{Initially estimated defaults.}
        \end{center}
    \end{column}
    \end{columns}

\end{frame}

%
% SLIDE
%

\begin{frame}[fragile]{Sensitivity Study - Parameters \& Initial Defaults}

    \begin{columns}
    \begin{column}{0.5\textwidth}
        \textbf{Parameters of Adaptive/Annealing PSO}
        \begin{itemize}
            \item number of iterations
            \item number of particles
            \item inertial weight
            \item acceleration individual optimum
            \item acceleration swarm optimum
            \item interpolation type
                \begin{itemize}
                    \item constant
                    \item cubic
                    \item exponential
                \end{itemize}
        \end{itemize}
    \end{column}

    \begin{column}{0.5\textwidth}
        \begin{center}
            \begin{tabular}{@{}lrrr@{}}
                \toprule
                & \multicolumn{3}{c}{Defaults}       \\
                \cmidrule(r){2-4}Parameter & SA & PSO & APSO\\
                \midrule
                iterations    & $1000$ &  $250$ &  $500$ \\
                step size     & $10$\% &        &        \\
                temperature   &   exp. &        &        \\
                particles     &        &  $100$ &  $100$ \\
                inertia       &        & $0.75$ & $0.75$ \\
                individual    &        &  $1.0$ &  $1.0$ \\
                swarm         &        &  $1.6$ &  $1.6$ \\
                interpolation &        &        & const. \\
                \bottomrule
            \end{tabular}
            \captionof{ }{Initially estimated defaults.}
        \end{center}
    \end{column}
    \end{columns}

\end{frame}
%
% SLIDE
%

\begin{frame}{Sensitivity Study - Simulated Annealing}
    \framesubtitle{varying \textbf{number of iterations}}

    \begin{columns}
    \begin{column}{0.8\textwidth}
        \includegraphics[width=\columnwidth]{../plots/initial_hyperparam_search__SA_iterations.pdf}
    \end{column}
    \end{columns}

    \begin{columns}
        \begin{column}{0.8\textwidth}
            $\implies$ optimal parameter range $\approx 1000-5000$ iterations
        \end{column}
        \begin{column}{0.1\textwidth}
            \tiny{\fbox{\textcolor{tabblue}{--} mean \; \colorbox{tabbluealpha}{ } std}}
        \end{column}
    \end{columns}

\end{frame}

%
% SLIDE
%

\begin{frame}{Sensitivity Study - Simulated Annealing}
    \framesubtitle{varying \textbf{step size} (in \% relative to search space)}

    \begin{columns}
    \begin{column}{0.8\textwidth}
        \includegraphics[width=\columnwidth]{../plots/initial_hyperparam_search__SA_step_size.pdf}
    \end{column}
    \end{columns}

    \begin{columns}
        \begin{column}{0.8\textwidth}
            $\implies$ optimal parameter range, step size $\approx 10$\%$-40$\% of search space
        \end{column}
        \begin{column}{0.1\textwidth}
            \tiny{\fbox{\textcolor{tabblue}{--} mean \; \colorbox{tabbluealpha}{ } std}}
        \end{column}
    \end{columns}

\end{frame}

%
% SLIDE
%

\begin{frame}{Sensitivity Study - Simulated Annealing}
    \framesubtitle{varying \textbf{temperature decay type}}

    \begin{columns}
    \begin{column}{0.8\textwidth}
        \includegraphics[width=\columnwidth]{../plots/initial_hyperparam_search__SA_temperature.pdf}
    \end{column}
    \end{columns}

    \begin{columns}
        \begin{column}{0.8\textwidth}
            $\implies$ negligible effect, keep as is
        \end{column}
        \begin{column}{0.1\textwidth}
            \tiny{\fbox{\textcolor{tabblue}{--} mean \; \colorbox{tabbluealpha}{ } std}}
        \end{column}
    \end{columns}
\end{frame}

%
% SLIDE
%

\begin{frame}{Sensitivity Study - Particle Swarm Optimization}
    \framesubtitle{varying \textbf{number of iterations}}

    \begin{columns}
    \begin{column}{0.8\textwidth}
        \includegraphics[width=\columnwidth]{../plots/initial_hyperparam_search__PSO_iterations.pdf}
    \end{column}
    \end{columns}

    \begin{columns}
        \begin{column}{0.8\textwidth}
            $\implies$ optimal parameter range $\approx 100-1000$ iterations
        \end{column}
        \begin{column}{0.1\textwidth}
            \tiny{\fbox{\textcolor{tabblue}{--} mean \; \colorbox{tabbluealpha}{ } std}}
        \end{column}
    \end{columns}
\end{frame}

%
% SLIDE
%

\begin{frame}{Sensitivity Study - Particle Swarm Optimization}
    \framesubtitle{varying \textbf{number of particles}}

    \begin{columns}
    \begin{column}{0.8\textwidth}
        \includegraphics[width=\columnwidth]{../plots/initial_hyperparam_search__PSO_n_particles.pdf}
    \end{column}
    \end{columns}

    \begin{columns}
        \begin{column}{0.8\textwidth}
            $\implies$ optimal parameter range $\approx 100-300$ particles
        \end{column}
        \begin{column}{0.1\textwidth}
            \tiny{\fbox{\textcolor{tabblue}{--} mean \; \colorbox{tabbluealpha}{ } std}}
        \end{column}
    \end{columns}
\end{frame}

%
% SLIDE
%

\begin{frame}{Sensitivity Study - Particle Swarm Optimization}
    \framesubtitle{varying \textbf{inertial decay factor}}

    \begin{columns}
    \begin{column}{0.8\textwidth}
        \includegraphics[width=\columnwidth]{../plots/initial_hyperparam_search__PSO_w.pdf}
    \end{column}
    \end{columns}

    \begin{columns}
        \begin{column}{0.8\textwidth}
            $\implies$ optimal parameter range $\approx 0.6 - 0.8$ inertial decay factor        \end{column}
        \begin{column}{0.1\textwidth}
            \tiny{\fbox{\textcolor{tabblue}{--} mean \; \colorbox{tabbluealpha}{ } std}}
        \end{column}
    \end{columns}
\end{frame}

%
% SLIDE
%

\begin{frame}{Sensitivity Study - Particle Swarm Optimization}
    \framesubtitle{varying \textbf{acceleration individual optimum}}

    \begin{columns}
    \begin{column}{0.8\textwidth}
        \includegraphics[width=\columnwidth]{../plots/initial_hyperparam_search__PSO_a_ind.pdf}
    \end{column}
    \end{columns}

    \begin{columns}
        \begin{column}{0.8\textwidth}
            $\implies$ negligible effect, keep as is
        \end{column}
        \begin{column}{0.1\textwidth}
            \tiny{\fbox{\textcolor{tabblue}{--} mean \; \colorbox{tabbluealpha}{ } std}}
        \end{column}
    \end{columns}
\end{frame}

%
% SLIDE
%

\begin{frame}{Sensitivity Study - Particle Swarm Optimization}
    \framesubtitle{varying \textbf{acceleration swarm optimum}}

    \begin{columns}
    \begin{column}{0.8\textwidth}
        \includegraphics[width=\columnwidth]{../plots/initial_hyperparam_search__PSO_a_neigh.pdf}
    \end{column}
    \end{columns}

    \begin{columns}
        \begin{column}{0.8\textwidth}
            $\implies$ negligible effect, keep as is
        \end{column}
        \begin{column}{0.1\textwidth}
            \tiny{\fbox{\textcolor{tabblue}{--} mean \; \colorbox{tabbluealpha}{ } std}}
        \end{column}
    \end{columns}
\end{frame}

%
% SLIDE
%

\begin{frame}{Sensitivity Study - Adaptive Particle Swarm Optimization}
    \framesubtitle{varying \textbf{number of iterations}}

    \begin{columns}
    \begin{column}{0.8\textwidth}
        \includegraphics[width=\columnwidth]{../plots/initial_hyperparam_search__APSO_iterations.pdf}
    \end{column}
    \end{columns}

    \begin{columns}
        \begin{column}{0.8\textwidth}
            $\implies$ negligible effect, $\geq 500$
        \end{column}
        \begin{column}{0.1\textwidth}
            \tiny{\fbox{\textcolor{tabblue}{--} mean \; \colorbox{tabbluealpha}{ } std}}
        \end{column}
    \end{columns}
\end{frame}

%
% SLIDE
%

\begin{frame}{Sensitivity Study - Adaptive Particle Swarm Optimization}
    \framesubtitle{varying \textbf{number of particles}}

    \begin{columns}
    \begin{column}{0.8\textwidth}
        \includegraphics[width=\columnwidth]{../plots/initial_hyperparam_search__APSO_n_particles.pdf}
    \end{column}
    \end{columns}

    \begin{columns}
        \begin{column}{0.8\textwidth}
            $\implies$ negligible effect, $\approx 200$
        \end{column}
        \begin{column}{0.1\textwidth}
            \tiny{\fbox{\textcolor{tabblue}{--} mean \; \colorbox{tabbluealpha}{ } std}}
        \end{column}
    \end{columns}
\end{frame}

%
% SLIDE
%

\begin{frame}{Sensitivity Study - Adaptive Particle Swarm Optimization}
    \framesubtitle{varying \textbf{inertial decay factor}}

    \begin{columns}
    \begin{column}{0.8\textwidth}
        \includegraphics[width=\columnwidth]{../plots/initial_hyperparam_search__APSO_w.pdf}
    \end{column}
    \end{columns}

    \begin{columns}
        \begin{column}{0.8\textwidth}
            $\implies$ optimal parameter range $\approx 0.7 - 0.8$ inertial decay factor
        \end{column}
        \begin{column}{0.1\textwidth}
            \tiny{\fbox{\textcolor{tabblue}{--} mean \; \colorbox{tabbluealpha}{ } std}}
        \end{column}
    \end{columns}
\end{frame}

%
% SLIDE
%

\begin{frame}{Sensitivity Study - Adaptive Particle Swarm Optimization}
    \framesubtitle{varying \textbf{acceleration individual optimum}}

    \begin{columns}
    \begin{column}{0.8\textwidth}
        \includegraphics[width=\columnwidth]{../plots/initial_hyperparam_search__APSO_a_ind.pdf}
    \end{column}
    \end{columns}

    \begin{columns}
        \begin{column}{0.8\textwidth}
            $\implies$ optimal parameter $1.0$ acceleration towards individual optimum.
        \end{column}
        \begin{column}{0.1\textwidth}
            \tiny{\fbox{\textcolor{tabblue}{--} mean \; \colorbox{tabbluealpha}{ } std}}
        \end{column}
    \end{columns}
\end{frame}

%
% SLIDE
%

\begin{frame}{Sensitivity Study - Adaptive Particle Swarm Optimization}
    \framesubtitle{varying \textbf{acceleration swarm optimum}}

    \begin{columns}
    \begin{column}{0.8\textwidth}
        \includegraphics[width=\columnwidth]{../plots/initial_hyperparam_search__APSO_a_neigh.pdf}
    \end{column}
    \end{columns}

    \begin{columns}
        \begin{column}{0.8\textwidth}
            $\implies$ optimal parameter $2.0$ acceleration towards swarm optimum.
        \end{column}
        \begin{column}{0.1\textwidth}
            \tiny{\fbox{\textcolor{tabblue}{--} mean \; \colorbox{tabbluealpha}{ } std}}
        \end{column}
    \end{columns}
\end{frame}


%
% SLIDE
%

\begin{frame}{Sensitivity Study - Adaptive Particle Swarm Optimization}
    \framesubtitle{varying \textbf{interpolation type}}

    \begin{columns}
    \begin{column}{0.8\textwidth}
        \includegraphics[width=\columnwidth]{../plots/initial_hyperparam_search__APSO_interpolation.pdf}
    \end{column}
    \end{columns}

    \begin{columns}
        \begin{column}{0.8\textwidth}
            $\implies$ negligible effect, keep as is (constant)
        \end{column}
        \begin{column}{0.1\textwidth}
            \tiny{\fbox{\textcolor{tabblue}{--} mean \; \colorbox{tabbluealpha}{ } std}}
        \end{column}
    \end{columns}
\end{frame}

%
% SLIDE
%

\begin{frame}[fragile]{Sensitivity Study - Conclusion}

    \begin{columns}
    \begin{column}{0.5\textwidth}
        \textbf{Simulated Annealing}
        \begin{itemize}
            \item The number of iterations is significantly higher w.r.t. (A)PSO due to each SA iteration being much cheaper in terms of computation. A step-size of $20$\% of the search space seems to suit most scenarios, anything above makes too big jumps. The temperature did not have any significant impact on the performance, thus it is kept as it initially was.
        \end{itemize}

    \end{column}

    \begin{column}{0.5\textwidth}
        \begin{center}
            \begin{tabular}{@{}lrrr@{}}
                \toprule
                & \multicolumn{3}{c}{Defaults}       \\
                \cmidrule(r){2-4}Parameter & SA & PSO & APSO\\
                \midrule
                iterations    & $5000$ & $400$ &  $500$ \\
                step size     & $20$\% &       &        \\
                temperature   &   exp. &       &        \\
                particles     &        & $200$ &  $200$ \\
                inertia       &        & $0.7$ & $0.75$ \\
                individual    &        & $1.0$ &  $1.0$ \\
                swarm         &        & $1.6$ &  $2.0$ \\
                interpolation &        &       & const. \\
                \bottomrule
            \end{tabular}
            \captionof{ }{Defaults after second sensitivity study.}
        \end{center}
    \end{column}
    \end{columns}

\end{frame}

\begin{frame}[fragile]{Sensitivity Study - Conclusion}

    \begin{columns}
    \begin{column}{0.5\textwidth}
        \textbf{Particle Swarm Optimization}
        \begin{itemize}
            \item In comparison to SA, the number of iterations of PSO are more than tenfold lower however at the expense of computing $200$ times more fitness evaluations per iterations. It seems that a slightly increased attraction towards the swarm optimum helps the convergence of the algorithm.
        \end{itemize}

    \end{column}

    \begin{column}{0.5\textwidth}
        \begin{center}
            \begin{tabular}{@{}lrrr@{}}
                \toprule
                & \multicolumn{3}{c}{Defaults}       \\
                \cmidrule(r){2-4}Parameter & SA & PSO & APSO\\
                \midrule
                iterations    & $5000$ & $400$ &  $500$ \\
                step size     & $20$\% &       &        \\
                temperature   &   exp. &       &        \\
                particles     &        & $200$ &  $200$ \\
                inertia       &        & $0.7$ & $0.75$ \\
                individual    &        & $1.0$ &  $1.0$ \\
                swarm         &        & $1.6$ &  $2.0$ \\
                interpolation &        &       & const. \\
                \bottomrule
            \end{tabular}
            \captionof{ }{Defaults after second sensitivity study.}
        \end{center}
    \end{column}
    \end{columns}

\end{frame}

\begin{frame}[fragile]{Sensitivity Study - Conclusion}

    \begin{columns}
    \begin{column}{0.5\textwidth}
        \textbf{Adaptive Particle Swarm Optimization}
        \begin{itemize}
            \item The adaptive weighting of the velocity components favors slighly more iterations and stronger attraction towards the swarm optimum for similar performance as PSO. The different interpolation methods used did not contribute significantly, thus the simplest is kept.
        \end{itemize}

    \end{column}

    \begin{column}{0.5\textwidth}
        \begin{center}
            \begin{tabular}{@{}lrrr@{}}
                \toprule
                & \multicolumn{3}{c}{Defaults}       \\
                \cmidrule(r){2-4}Parameter & SA & PSO & APSO\\
                \midrule
                iterations    & $5000$ & $400$ &  $500$ \\
                step size     & $20$\% &       &        \\
                temperature   &   exp. &       &        \\
                particles     &        & $200$ &  $200$ \\
                inertia       &        & $0.7$ & $0.75$ \\
                individual    &        & $1.0$ &  $1.0$ \\
                swarm         &        & $1.6$ &  $2.0$ \\
                interpolation &        &       & const. \\
                \bottomrule
            \end{tabular}
            \captionof{ }{Defaults after second sensitivity study.}
        \end{center}
    \end{column}
    \end{columns}

\end{frame}

\section{Benchmark Functions}

\begin{frame}[fragile]{Benchmark Functions - Overview}
    \begin{itemize}
        \item test functions too easy $\implies$ bad for benchmark, good for tuning of parameters, 
        \item taken from \href{https://arxiv.org/abs/2202.04606}{\textbf{New Hard Benchmark Functions for Global Optimization}}
        \item want more challenging minimization problems for benchmark
            \begin{itemize}
                \item multi-modality $\implies$ Layeb04, Layeb05, Layeb18
                \item steep and high-frequent convexities $\implies$ Layeb09
                \item very hard for most algorithms\footnotemark[1] $\implies$ Layeb12 \item hard for most algorithms\footnotemark[1] $\implies$ Layeb14
            \end{itemize}
    \end{itemize}

    \footnotetext[1]{\tiny{Claimed by the author of the paper.}\smallskip}
\end{frame}

\begin{frame}[fragile]{Benchmark Functions - Contour Plot}
    \begin{columns}
    \begin{column}{0.8\textwidth}
        \includegraphics[width=\columnwidth]{../plots/overview_benchmarkfunction_contour.pdf}
    \end{column}
    \end{columns}
    \tiny{Note: White plus symbols denote global optima}
\end{frame}

\begin{frame}[fragile]{Benchmark Functions - Surface Plot}
    \begin{columns}
    \begin{column}{0.8\textwidth}
        \includegraphics[width=\columnwidth]{../plots/overview_benchmarkfunction_surface.pdf}
    \end{column}
    \end{columns}
    \tiny{Note: best viewed in full-screen!}
\end{frame}

\section{Benchmark}

\begin{frame}{Benchmark - Overview}

    \textbf{Assess goodness} of competing optimizers on different minimization problems
    \begin{enumerate}
        \item perform $20$ stochastic \textbf{independent retires} of same experiments\\
            $\implies$ reduce fluctuations due to randomness and noise
        \item track \textbf{fitness vs. number of fitness calls}
            \begin{itemize}
                \item fitness: $\exp({\tiny{\text{opt}_\text{best}}} - {\tiny{\text{opt}_\text{found}}}) \in \lbrack 0,1\rbrack$
                \item fitness calls: number of evaluations of function to be minimized
            \end{itemize}
            $\implies$ measure \textbf{accuracy} and \textbf{efficiency}
    \end{enumerate}

    \medskip

    Benchmark Plot Legend:
    \begin{itemize}
        \item \textbf{dots}: fitness of encountered value vs. number of fitness evaluations
        \item \textbf{crosses}: best dot per independent retry
        \item \textbf{dashed line}: mean of crosses
        \item \textbf{opaque band}: standard deviation of crosses
    \end{itemize}
\end{frame}

\begin{frame}[fragile]{Benchmark - Simulated Annealing}
    \begin{columns}
    \begin{column}{0.8\textwidth}
        \includegraphics[width=\columnwidth]{../plots/benchmark__SA.pdf}
    \end{column}
    \end{columns}
\end{frame}

\begin{frame}[fragile]{Benchmark - Particle Swarm Optimization}
    \begin{columns}
    \begin{column}{0.8\textwidth}
        \includegraphics[width=\columnwidth]{../plots/benchmark__PSO.pdf}
    \end{column}
    \end{columns}
\end{frame}

\begin{frame}[fragile]{Benchmark - Adaptive Particle Swarm Optimization}
    \begin{columns}
    \begin{column}{0.8\textwidth}
        \includegraphics[width=\columnwidth]{../plots/benchmark__APSO.pdf}
    \end{column}
    \end{columns}
\end{frame}

\section{Conclusion}

\begin{closingframe}{Conclusion}

    \textbf{Simulated Annealing}:
    \begin{itemize}
        \item Three out of six benchmark functions were well above $90$\% fitness, which speaks for the capabilities of SA. It clearly struggled with \texttt{Layeb04} and \texttt{Layeb18} but got the fitness above $70$\%. However \texttt{Layeb14} was a complete disaster and the algoritm was not abe to find the optimum. Due to the exponential nature of the fitness, even half good solutions are portrayed as quite poorly. This fitness measure was chosen, because the other plots then would have been completely filled. In comparison to (A)PSO, SA is quite an efficient algorithm with only one fitness evaluation per iteration and as one can see from the benchmark plots, it finds most optima relatively early.
    \end{itemize}
\end{closingframe}

\begin{closingframe}{Conclusion}
    \textbf{Particle Swarm Optimization}:
    \begin{itemize}
        \item PSO performed better than SA as five out of six benchmark functions achieve a fitness of over $80$\%. Similar to SA it struggled slightly with \texttt{Layeb04} and \texttt{Layeb18}. Also PSO has not reached any significant fitness on the \texttt{Layeb14} optimization problem. It is quite a computationally expensive algorithm with each iteration costing as much fitness evaluations as it has particles. Nonetheless, this trade-off comes with increased optimizing capabilities.
    \end{itemize}
\end{closingframe}

\begin{closingframe}{Conclusion}
    \textbf{Adaptive Particle Swarm Optimization}:
    \begin{itemize}
        \item The hypothesis that APSO would outperform SA and PSO on \texttt{Layeb14} due to taking the best of both worlds, had to be discarded unfortunately. This hypothesis however still holds for the other five benchmark functions, which we deem as partial success. It is equally efficient as PSO but outperforms it in every benchmark. The benchmark plot is visually very interesting as one clearly sees the three phases. Among these phases, a visual trend is noticable, especially in \texttt{Layeb05} and \texttt{Layeb18} where with increasing phases the algorithm clearly finds more and better solutions to the optimization problems. Concluding, one can say that on these problems, APSO clearly outperforms PSO with almost no additional cost, thus the hypothesis at least holds partially. Moreover, further experimentation with the interpolation types and different phases could yield even more success.
    \end{itemize}
\end{closingframe}

\end{document}
